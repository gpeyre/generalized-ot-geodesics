\documentclass[11pt]{article}
\usepackage{amsmath,amssymb,geometry}
\geometry{margin=1in}

\title{Generalized OT Geodesics: Short Mathematical Summary}
\author{}
\date{}

\begin{document}
\maketitle

\section*{Problem Setup}
Let $X_0, X_1 \in \mathbb{R}^{d \times n}$ be point clouds with $n$ particles in dimension $d$.
We optimize a discrete path
\[
X = (X_t)_{t=0}^{T-1}, \quad X_t \in \mathbb{R}^{d \times n},
\]
where $X_t(:,i)$ denotes particle $i$ at time $t$.

\section*{Energy}
We use the objective
\[
\mathcal{E}(X) =
W_2^2(X_{T-1}^{\mathrm{path}}, X_1)
+ \frac{\gamma}{n(T-1)} \sum_{t=1}^{T-1}\sum_{i=1}^n \phi(X_t, X_t(:,i)) \|X_t(:,i)-X_{t-1}(:,i)\|_2^2.
\]
Here $\gamma > 0$ controls regularity and trajectory smoothness.
The initial state is imposed as a hard constraint:
\[
X_0^{\mathrm{path}} = X_0.
\]
Hence only $\{X_t\}_{t=1}^{T-1}$ are optimized.

\section*{Discrete Wasserstein Term}
For clouds $Y,Z \in \mathbb{R}^{d \times n}$ with uniform weights, the squared 2-Wasserstein cost is
\[
W_2^2(Y,Z)=\min_{\Pi \in \mathcal{U}(a,b)} \sum_{i,j}\Pi_{ij}\|y_i-z_j\|_2^2,
\]
with $a=b=\frac{1}{n}\mathbf{1}$ and $\mathcal{U}(a,b)$ the transport polytope.
In the implementation, this is computed with POT.

\section*{Choices of $\phi$}
\begin{itemize}
  \item Classical OT-like kinetic term: $\phi \equiv 1$.
  \item Transformer-OT inspired variant:
  \[
  \phi(X_t, \cdot) = \sum_{i,j} \exp\!\left(-\frac{\|X_t(:,i)-X_t(:,j)\|_2^2}{2\beta^2}\right),
  \]
  which is independent of the query position and uses a temperature $\beta>0$
  (default $\beta=1$).
\end{itemize}

\section*{Optimization}
Any feasible initialization can be used for $t \ge 1$ while keeping $X_0^{\mathrm{path}}=X_0$.
In the notebook example we use the trivial feasible path
\[
X_t^{(0)} = X_0 \quad \text{for all } t=0,\dots,T-1.
\]
Then we minimize $\mathcal{E}$ with L-BFGS in PyTorch.
The constrained structure is enforced by parameterizing and optimizing only the tail slices
$X_1,\dots,X_{T-1}$, while concatenating $X_0$ as a fixed first slice at every iteration.

To improve robustness, the code also provides a continuation strategy over decaying
kinetic weights $\gamma$:
\[
\gamma_1 > \gamma_2 > \cdots > \gamma_K,
\]
where each stage is warm-started from the previous optimized path.
This is exposed in the API as \texttt{optimize\_with\_gamma\_schedule(...)}.

For reference, an alternative unconstrained warm-start is linear interpolation:
\[
X_t^{(0)} = (1-\alpha_t)X_0 + \alpha_t X_1, \quad \alpha_t=\frac{t}{T-1}.
\]
At each closure call, POT provides an OT plan for endpoint terms, and PyTorch evaluates
the plan-weighted cost to backpropagate through point locations.

\section*{Current Numerical Studies}
In addition to the main 2D demo, a dedicated parameter-study notebook evaluates:
\begin{itemize}
  \item \textbf{Impact of $\beta$:} $X_0$ is sampled from a single Gaussian and $X_1$
  from a two-Gaussian mixture; interpolations are computed for four temperatures.
  \item \textbf{Impact of variance:} $X_0$ and $X_1$ are single Gaussians with different means
  and shared variance $\sigma^2$; interpolations are compared across several $\sigma$ values
  at fixed $\beta=1$.
\end{itemize}
Both studies use $n=50$ particles and a 5-stage geometric schedule
$\gamma \in [1,10^{-4}]$, and display final interpolating paths.

\end{document}
